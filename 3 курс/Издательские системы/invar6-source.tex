% "Инвариативная самостоятельная работа № 6"

\documentclass[a4paper,12pt]{article} % тип документа -- лсит А4 с 12 шрифтом текста

\usepackage[T2A]{fontenc}			% кодировка
\usepackage[utf8]{inputenc}			% кодировка исходного текста
\usepackage[english,russian]{babel}	% локализация и переносы

\usepackage{amsmath,amsfonts,amssymb,amsthm,mathtools} % математические символы, формулы и т. д.

\usepackage{wasysym}

\usepackage{hyperref}

\author{Беленко А.В., 1 гр. 2 подгр.} % имя автора
\title{Таблица команд} % название документа
\date{\today} % дата создани документа

\begin{document}
\maketitle
\newpage
\section*{Команды форматирования в \LaTeX}
\begin{tabular}{ l | r }
  \textbf{Назначение команды} & \textbf{Вид (написание) команды} \\ \hline
  самый маленький			  & tiny \\ \hline
  меньше меньшего			  & scriptsize \\ \hline
  меньший					  & footnotesize \\ \hline
  маленький					  & small \\ \hline
  обычный					  & normalsize \\ \hline
  большой					  & large \\ \hline
  больший					  & Large \\ \hline
  больше большего			  & LARGE \\ \hline
  очень большой				  & huge \\ \hline
  самый большой				  & Huge \\ \hline
  создать сноску			  & footnote \\ \hline
  гиперссылка				  & href \\ \hline
  выравнивание по центру	  & begin{center}, end{center} \\ \hline
  выравнивание по правому краю & begin{flushright}, end{flushright} \\ \hline
  выравнивание по левому краю & begin{flushleft}, end{flushleft} \\ \hline
  нумерованный список 		  & begin{enumerate}, item, end{enumerate} \\ \hline
  маркированный список		  & begin{itemize}, item, end{itemize} \\ \hline
  сноска					  & footnote
\end{tabular}
\end{document}