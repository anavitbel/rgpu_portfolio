% "Лабораторная работа № 8"

\documentclass[a4paper,12pt]{article} % тип документа

% report, book

% Русский язык

\usepackage[T2A]{fontenc}			% кодировка
\usepackage[utf8]{inputenc}			% кодировка исходного текста
\usepackage[english,russian]{babel}	% локализация и переносы


% Математика
\usepackage{amsmath,amsfonts,amssymb,amsthm,mathtools} 


\usepackage{wasysym}

\usepackage{hyperref}

%Заговолок
\author{Беленко А.В., 1 гр. 2 подгр.}
\title{Создание матриц средствами \LaTeX}
\date{\today}


\begin{document} % начало документа

\maketitle
\newpage


\begin{flushleft}

\section*{Пример 1. Умножение матрицы на число}
Дано: Матрица
$ A=\begin{pmatrix}
1 & 2 & 3 \\
4 & 5 & 6
\end{pmatrix} $. \\
Число $k=2$.\\
Найти:\\
Произведение матрицы на число: $A \times k=B$
Решение:\\
Для того чтобы умножить матрицу $A$ на число $k$ нужно каждый элемент матрицы $A$ умножить на это число.\\
Таким образом, произведение матрицы $A$ на число $k$ есть новая матрица: \\
$$ B=2 \times A=2 \times \begin{pmatrix}
1 & 2 & 3 \\
4 & 5 & 6
\end{pmatrix}=\begin{pmatrix}
2 & 4 & 6 \\
8 & 10 & 12
\end{pmatrix}$$ \\
Ответ: $B=\begin{pmatrix}
2 & 4 & 6 \\
8 & 10 & 12
\end{pmatrix}$


\section*{Пример 2. Умножение матриц}
Дано: Матрица
$ A=\begin{pmatrix}
2 & 3 & 1 \\
-1 & 0 & 1
\end{pmatrix} $; \\
Матрица
$ B=\begin{pmatrix}
2 & 1 \\
-1 & 1 \\
3 & -2
\end{pmatrix} $ \\
Найти:\\
Произведение матриц: $A \times B=C$ \\
$C-?$ \\
Решение:\\
Каждый элемент матрицы $C=A \times B$, расположенный в $i$-й строке и $j$-ом столбце, равен сумме произведений элементов $i$-й строки матрицы $A$ на соответствующие элементы $j$-го столбца матрицы $B$. Строки матрицы $A$ умножаем на столбцы матрицы $B$ и получаем:\\
$$C=A \times B=\begin{pmatrix}
2 & 3 & 1 \\
-1 & 0 & 1
\end{pmatrix} \times \begin{pmatrix}
2 & 1 \\
-1 & 1 \\
3 & -2
\end{pmatrix} =$$\\
$$=\begin{pmatrix}
2 \times 2+3 \times (-1)+1 \times 3 & 2 \times 1+3 \times 1+1 \times (-2) \\
-1 \times 2+0 \times (-1)+1 \times 3 & -1 \times 1+0 \times 1+1 \times (-2)
\end{pmatrix}$$
$$C=A \times B=\begin{pmatrix}
4 & 3 \\
1 & -3
\end{pmatrix}
$$
Ответ: $B=\begin{pmatrix}
4 & 3 \\
1 & -3
\end{pmatrix}$


\section*{Пример 3. Транспонирование матрицы}
Дано: Матрица
$ A=\begin{pmatrix}
7 & 8 & 9 \\
1 & 2 & 3
\end{pmatrix} $. \\
Найти:\\
Найти матрицу транспонированную данной.\\
$A^T=?$\\
Решение:\\
Транспонирование матрицы $A$ заключается в замене строк этой матрицы ее столбцами с сохранением их номеров. Полученная матрица обозначается через $A^T$\\
$$ A=\begin{pmatrix}
7 & 8 & 9 \\
1 & 2 & 3
\end{pmatrix} \Rightarrow A^T=\begin{pmatrix}
7 & 1 \\
8 & 2 \\
9 & 3
\end{pmatrix}$$\\
Ответ: $ A^T=\begin{pmatrix}
7 & 1 \\
8 & 2 \\
9 & 3
\end{pmatrix}$

\section*{Пример 4. Обратная матрица}
Дано: Матрица
$ A=\begin{pmatrix}
2 & -1 \\
3 & 1
\end{pmatrix} $. \\
Найти:\\
Найти обратную матрицу для матрицы $A$.\\
$A^{-1}-?$\\
Решение:\\
Находим $det A$ и проверяем $det A \neq 0$:\\
$$det A=\begin{vmatrix}
2 & -1 \\
3 & 1
\end{vmatrix}=2 \times 1-3 \times (-1) = 5$$
$$det A = 5 \neq 0.$$ \\
Составляем вспомогательную таблицу $A^V$ из алгебраических дополнений $A_{ij}$: $A^V=\begin{pmatrix}
1 & -3 \\
1 & 2
\end{pmatrix}$\\
Транспонируем матрицу $A^V$: $(A^V)^T=\begin{pmatrix}
1 & 1 \\
-3 & 2
\end{pmatrix}$\\
Каждый элемент полученной матрицы делим на $det A$:\\
$$A^{-1}=\frac{1}{det A}(A^V)^T=\frac{1}{5} \times \begin{pmatrix}
1 & 1 \\
-3 & 2
\end{pmatrix} = \begin{pmatrix}
\frac{1}{5} & \frac{1}{5} \\
-\frac{3}{5} & \frac{2}{5}
\end{pmatrix}$$\\
Ответ: $A^{-1}=\begin{pmatrix}
\frac{1}{5} & \frac{1}{5} \\
-\frac{3}{5} & \frac{2}{5}
\end{pmatrix}$

\end{flushleft}
\end{document} % конец документа